\documentclass[9pt]{beamer}
\usepackage[utf8]{inputenc}
\usepackage{txfonts}
\usepackage[english]{babel}
\usepackage{xcolor}
\usetheme{AnnArbor}
\usecolortheme{beaver}

\setbeamertemplate{headline}{}
% \setbeamertemplate{frametitle}{\insertframetitle}
\setbeamertemplate{navigation symbols}{}

\setbeamertemplate{itemize item}{\scriptsize\raise1.25pt\hbox{\donotcoloroutermaths$\blacktriangleright$}}
\setbeamertemplate{itemize subitem}{\tiny\raise1.5pt\hbox{\donotcoloroutermaths$\blacktriangleright$}}
\setbeamertemplate{itemize subsubitem}{\tiny\raise1.5pt\hbox{\donotcoloroutermaths$\blacktriangleright$}}
\setbeamertemplate{enumerate item}{\insertenumlabel.}
\setbeamertemplate{enumerate subitem}{\insertenumlabel.\insertsubenumlabel}
\setbeamertemplate{enumerate subsubitem}{\insertenumlabel.\insertsubenumlabel.\insertsubsubenumlabel}
\setbeamertemplate{enumerate mini template}{\insertenumlabel}

% \setbeamertemplate{itemize items}[square]
% \setbeamertemplate{items}[square]



\newcommand{\bluemph}[1]{\structure{\emph{#1}}}
\newcommand{\redemph}[1]{\alert{\emph{#1}}}
\newcommand{\bluebf}[1]{\structure{\textbf{#1}}}
\newcommand{\redbf}[1]{\alert{\textbf{#1}}}

\newcommand\denote[1]{\llbracket #1 \rrbracket}
\newcommand\fesi{Fe-Si}

%  Structure
\newenvironment{remark}{\footnotesize \begin{description}\item[\emph{Remark}:]}{\end{description}}

\title{GT Coq}
\author{Thomas Braibant}
\institute[Inria]{Inria}
\date[11/2013]{11/2013}

\setbeamercovered{transparent}
%\setbeamerfont{frametitle}{size={\normalsize}}

% \usepackage[T1]{fontenc}
\usepackage{amsmath}
\usepackage{amssymb}
\usepackage{amsthm}
\usepackage{mathpartir}

\usepackage{listings}
\usepackage{graphicx}
\definecolor{ltblue}{rgb}{0,0.4,0.4}
\definecolor{dkblue}{rgb}{0,0.1,0.6}
\definecolor{dkgreen}{rgb}{0,0.4,0}
\definecolor{dkviolet}{rgb}{0.3,0,0.5}
\definecolor{dkred}{rgb}{0.5,0,0}
\usepackage{lstcoq}
\usepackage{lstocaml}
\usepackage{tikz}

\begin{document}
% \newcommand \blue[1]{{\color{red!80!black}{#1}}}
\newcommand \orange[1]{{\color{orange}{#1}}}
% \newcommand \red[1]{{\color{red}{#1}}}
% \newcommand \grey[1]{{\color{gray}{#1}}}
% \newcommand \green[1]{{\color{violet}{#1}}}
% \newcommand \white[1]{{\color{white}{#1}}}

\newcommand\parenthesis[1] {
  \begin{flushright}
    {\scriptsize \redemph{{{{ #1}}}}}
  \end{flushright}

}

\newcommand\plan[2]
{
\begin{frame}[plain]
  \begin{center}
    {\Huge  \sc #1} \\

    \vspace{1cm}

    #2
\end{center}
\end{frame}
}

\begin{frame}
  \center 
  \titlepage
\end{frame} 

\begin{frame}{Problems}
  \begin{itemize}
  \item No centralized repository of packages
  \item No unified build system
  \item Monolithic packages (e.g. ssreflect, compcert, CoLoR, \dots)
  \item No easy way to have multiple Coq installs \\
    \parenthesis{symlink, make install, and so on}
  \end{itemize}
\end{frame}


\plan{Plan A}{\textbf{Implement a package manager from scratch}}

\begin{frame}
  \frametitle{Plan A}
  \begin{itemize}
  \item Time consuming
  \item Error prone
  \item Not ``science''
  \end{itemize}
\end{frame}

\plan{Plan B}{\textbf{Forking an exisiting package manager: opam}}

\begin{frame}
  \title{Forking opam}
  \begin{itemize}
  \item opam knows of (ocaml) \alert{'compilers'} and 'packages'
    \pause
  \item search and replace:
    \begin{itemize}
    \item opam with cpam
    \item ocaml with coq
    \end{itemize}
    \pause
  \item What if the OCaml version of the system changes?
  \item What about dependencies between Coq packages (plugins) and OCaml  packages?
  \end{itemize}
\end{frame}


\plan{Plan C}{\textbf{Modify an existing package manager: opam}}

\begin{frame}[fragile]
  \frametitle{Opam primer}
  \begin{itemize}
  \item Each 'state' is stored in one {\verb!$opam/$switch!}
  \item Each 'state' has its own OCaml compiler and set of packages
  \item {\tt opam switch 4.01.0} will:
    \begin{itemize}
    \item create a new switch if needed
    \item install the ocaml compiler version 4.01.0
      \pause
    \item update the configuration env. if the switch already exists
    \end{itemize}
    \pause
  \item {\tt opam install foobar} will:
    \begin{itemize}
    \item lookup for the available versions of package {\tt foobar};
    \item select the highest version available for the current compiler; 
    \item update other packages if needed
    \end{itemize}
  \end{itemize}
  
  \pause
  
  \begin{center}
    Powerful system to resolve dependencies between packages
  \end{center}
\end{frame}

\begin{frame}
  \frametitle{Prototype}
  
  New commands for opam. 
  \begin{itemize}
  \item {\tt opam coq install coq.8.4.2} will:
    \begin{itemize}
    \item create a new opam switch named {\tt myocamlversion--coq.8.4.2}
    \item install Coq 8.4pl2 in this switch (with its dependency on camlp5)
    \end{itemize}
    \pause
  \item {\tt opam coq switch coq.HoTT} will:
    \begin{itemize}
    \item browse the existing switches to find one with Coq HoTT; 
      \pause
    \item if there is exactly one, will 'switch' to it;
    \item if there is more than one, will tell the user to select one;
    \item if there is none, will install one (with a fresh a OCaml compiler)
    \end{itemize}
    \pause
  \item {\tt opam coq list} will: 
    \begin{itemize}
    \item show the available installs of Coq (similiar to {\tt opam switch})
    \end{itemize}
  \end{itemize}
\end{frame}

\begin{frame}[fragile]
  \frametitle{One word}
  
  \begin{itemize}
  \item Coq is a \alert{regular} opam package
  \item Coq packages are \alert{regular} opam packages
  \item The front-end uses \alert{regular} opam switches
  \item The front-end uses \alert{regular} opam 'pins' (to pin the installed version of Coq)
  \end{itemize}
  
\end{frame}

\begin{frame}[fragile]
  \frametitle{Coq 8.4pl2}

\lstset{basicstyle=\footnotesize, frame=single}

\begin{lstlisting}
archive: "http://coq.inria.fr/distrib/V8.4pl2/files/coq-8.4pl2.tar.gz"    
\end{lstlisting}

  \begin{columns}
\column{0.45 \linewidth}

  \begin{lstlisting}
opam-version: "1.1"
patches: ["configure.patch"]
build: [
  [ "./configure" 
     "-configdir" "%{lib}%/coq/config"
     "-mandir" "%{man}%"
     "-docdir" "%{doc}%"
     "--prefix" "%{prefix}%"
     "--usecamlp5"
     "--coqide" "no"
#     "-opt"
  ]
  [ "make" "-j4" "world"]
  [ "make" "install"]
]
depends: ["camlp5"]    
  \end{lstlisting}
    \column{0.35\linewidth}
    \begin{itemize}
    \item camlp5 by default
    \item opt compilers?
    \item set of patch
    \end{itemize}
  \end{columns}
\end{frame}

\begin{frame}[fragile]
  \frametitle{Coq 8.4pl2+mtac}
  \begin{columns}
\column{0.4 \linewidth}
\lstset{basicstyle=\footnotesize}
  \begin{lstlisting}
opam-version: "1.1"
patches: ["configure.patch" "mtac-1.1.patch"]
build: [
  [ "./configure" 
     "-configdir" "%{lib}%/coq/config"
     "-mandir" "%{man}%"
     "-docdir" "%{doc}%"
     "--prefix" "%{prefix}%"
     "--usecamlp5"
     "--coqide" "no"
#     "-opt"
  ]
  [ "make" "-j4" "world"]
  [ "make" "install"]
]
depends: ["camlp5"]    
  \end{lstlisting}
    \column{0.35\linewidth}
    \begin{itemize}
      \item One line of difference!
    \end{itemize}
  \end{columns}
\end{frame}

\begin{frame}[fragile]
  \frametitle{ssreflect}

\lstset{basicstyle=\footnotesize}

  \begin{lstlisting}
archive:"http://ssr.msr-inria.inria.fr/FTP/ssreflect-1.5rc1.tar.gz"
checksum: "c08130242ea2cfd1cb4ae8754fa411fe"
  \end{lstlisting}
  \begin{columns}
\column{0.4 \linewidth}
  \begin{lstlisting}
opam-version: "1.1" 
maintainer: "thomas.braibant@gmail.com"
build: [
  [make]
  [make "install"]
]
depends: [ "coq" { >= "8.4.2"}]
  \end{lstlisting}
    \column{0.35\linewidth}
    \begin{itemize}
      \item Version constraints
    \end{itemize}
  \end{columns}
\pause
\begin{center}
  In practice, in Coq, all package constraints should use 'equals'!
\end{center}
\end{frame}

\begin{frame}
  \frametitle{Current state}
  \begin{center}
    \begin{tabular}{l}
      HoTT.stable	\\
      aac\_tactics.0.4	 \\
      compcert.2.0+ia32-linux \\
      containers.2010	 \\
      coq.8.4.2+mtac	\\
      coq.8.4.2	\\
      coq.8.4.3dev	\\
      coq.8.5dev	\\
      coq.HoTT+stable	\\
      counting.2010	\\
      cybele.1.2dev	\\
      ergo.2010	        \\
      flocq.2.2.0	\\
      interval.0.16.1	\\
      mathcomp.1.5	\\
      ssreflect.1.5  
    \end{tabular}
  \end{center}
\end{frame}

\begin{frame}
  \frametitle{Questions}
  
  \begin{itemize}
  \item \alert{Broken} packages
    \begin{itemize}
    \item Ergo (from contribs) does not compile out of the box!
    \item CoLoR has no install target
    \end{itemize}
    \pause
  \item Configuration options? (one of a kind problem for CompCert)
    \begin{itemize}
    \item One package per compcert set of options?
    \end{itemize}
    \pause
  \item Semantic versioning? ({\tt semver.org}) \\
    \begin{quote}
      Given a version number MAJOR.MINOR.PATCH+tag, increment the:
      \begin{enumerate}
      \item MAJOR version when you make incompatible API changes;
      \item MINOR version when you add functionality in a
        backwards-compatible manner;
      \item PATCH version when you make backwards-compatible bug
        fixes.
      \end{enumerate}
    \end{quote}
    \pause
  \item How to deal with git branches as releases? (HoTT)
  \item camlp5/camlp4?
  \item coq\_makefile, uninstall targets?
  \end{itemize}
\end{frame}

\begin{frame}
  \frametitle {Perspectives}

  \begin{itemize}
  \item Fine grained packages
  \item Stripped down Coq (libraries and plugins as separate packages)
  \item Download statistics
  \item Opam2Web (to list packages)
  \item Ocamlot (CI/test infrastructure)
  \item SDK for Windows, OS X (Vagrant)?
  \end{itemize}
\end{frame}

\plan{Plan D}{\textbf{Package Coq specific commands as an opam package}}


\plan{TL; DR}{\textbf{    
\begin{tabular}{l}
      OPAM \\
      ONLY\\
      DOES \\
      \alert{EVERYTHING}
    \end{tabular}
}    

\tiny{thanks OCamlPro}
}

\plan{One more thing}{
  \textbf{Survey about Coq}

  \vspace{1em}

  \begin{itemize}
  \item Know the users (researchers, teachers, students, numbers)
  \item Know what they do (math, pl, production)
  \item Know how they work (ide, how much time, tactics)
  \item Know what library/plugin/contrib they use (are contribs used)
  \item Know what they find important (doc, perf, backward compatibility, new features)
  \item Guide development
  \end{itemize}
}


\end{document}
